\documentclass[12pt]{ctexart}
\usepackage[top=2cm, bottom=2cm, left=2cm, right=2cm]{geometry}
\usepackage{algorithm}
\usepackage{algorithmicx}
\usepackage{algpseudocode}
\usepackage{amsmath}
\usepackage{caption}
\usepackage{graphicx, subfig}
\setcounter{algorithm}{0}
\floatname{algorithm}{Greedy}
\title{计算机算法设计与分析\\ 第三次作业}
\author{201818013229056 苗天昊}
\begin{document}

\begin{figure}
\centering
\includegraphics[scale=0.12]{ucas_logo.pdf}
\end{figure}

\maketitle
\section{Greedy Algorithm}
Given a list of $n$ natural numbers $d_1, d_2,...,d_n$, show how to decide in polynomial time whether there exists an undirected graph $G = (V, E)$ whose node degrees are precisely the numbers $d_1, d_2,...,d_n$. G should not contain multiple edges between the same pair of nodes, or “ loop” edges with both endpoints equal to the same node.
\par 问题分析与证明:
\par 先考虑最小情况。
\par 1)当图G没有顶点时,不存在顶点的度。当顶点度的集合为空时,可以构成一个空图。
\par 2)当图G只有一个顶点时,顶点度的集合非空,并且只有一个元素,其值为0时可以构成一个顶点的图。
\par 3)当图G有两个顶点时,顶点度的集合为\{$d_1, d_2$\}。不妨假设$d_1$是较大值,如果$d_1$大于2-1=1,则肯定无法构成无向图。并且$d_1$值非负,只要顶点的度有一个值为负数,则该顶点度数集肯定不能构成无向图。所以$d_1$的取值只有两种可能0和1,如果$d_1=1$,所以顶点之间有一条边,将度数为$d_1$的顶点$V_1$删除,则$d_2$对应的顶点$V_2$少了一条与$V_1$相连的边。其子问题转化为从\{$d_2-1$\}的顶点度数集中判断其能否组成无向图。$d_2$的值只有当且仅当1的时候可以在$d_1=1$的情况下构成无向图。如果$d_1=0$,则其子问题转化为从\{$d_2$\}的顶点度数集中判断其能否组成无向图,当切仅当$d_2$的值为0时,可以构成无向图。
\par 现考虑一般情况,给出有$n$个顶点的顶点度数集合\{$d_1, d_2,...,d_n$\},同样,我们将该顶点度数集进行排序,不妨假设$d_1, d_2,...,d_n$是经过排序的,并且$d_1>=d_2>=...>=d_n$,令$d_1$的值为k,即$d_1$对应的顶点$V_1$有k条边,现将顶点$V_1$删除,有边连接的各顶点度数减1,其子问题就是从$d_2-1,d_3-1,.,d_k-1,d_{k+1}..,d_n$中判断其能否构成无向图。
\par 需要证明,当前度数最大(其值为k)的点为$V_1$,其一定与度数序列中的后k个顶点有边。假设存在一个顶点$V_i$$(2\leq i \leq k)$,其与$V_1$不存在边连接。因为$V_1$的度数为$k$,所以一定存在个值$j>k$,$V_j$与$V_1$存在边。由于$d_i>d_k>d_j$,所以一定存在一个顶点$V_k$与$V_i$存在边,而与$V_j$不存在边,即一定存在边($V_i$, $V_k$)。于是,可以将边($V_1$, $V_j$)和$V_i$, $V_k$)删除,添加边($V_1$, $V_i$)和($V_j$, $V_k$),变化后顶点度数集合不变,可以依次调整边,使度数为$d_1$的点与$d_2,d_3,...,d_k$存在边。

\begin{algorithm}[h]
    \begin{algorithmic}[0] %每行显示行号
    \caption{Greedy 1}
    \State sort $d_1, d_2,...,d_n$ as d[n] //降序排列$d_1, d_2,...,d_n$
    \If{$d[0]$ < 0 || $d[0]$ > n-1}
    \State \Return False
    \EndIf
    \For{$i=0$ to $n-1$}
    	\State $k = d[0]$
	\If{$d[i] < 0$ or $d[0]$ > n-1}
		\State \Return False
	\EndIf
    	\For{$j = i+1$ to $k-1$}
    		\State $d[j] = d[j] - 1$
	\EndFor
    \EndFor
    \If{$d[n-1]$ == 0}
    \State \Return True
    \Else 
    \State \Return False
    \EndIf
    \end{algorithmic}
\end{algorithm}
\par 时间复杂度分析:该算法首先需要对集合进行排序,其时间复杂度至少为O($log(n)$)。在计算顶点的度数时,用了两层循环,$n$个节点的度数最大为$n-1$,则两层循环的时间复杂度为O($n^2$)。所以综合来看,该算法的时间复杂度为O($n^2$)。
\newpage
\section{Greedy Algorithm}
There are $n$ distinct jobs, labeled $J1, J2, · · · , Jn$, which can be performed completely independently of one another. Each jop consists of two stages: first it needs to be preprocessed on the supercomputer, and then it needs to be finished on one of the PCs. Let’s say that job $J_i$ needs $p_i$ seconds of time on the supercomputer, followed by $f_i$ seconds of time on a PC. Since there are at least $n$ PCs available on the premises, the finishing of the jobs can be performed on PCs at the same time. However, the supercomputer can only work on a single job a time without any interruption. For every job, as soon as the preprocessing is done on the supercomputer, it can be handed off to a PC for finishing.
\par Let’s say that a schedule is an ordering of the jobs for the supercomputer, and the completion time of the schedule is the earlist time at which all jobs have finished processing on the PCs. Give a polynomial-time algorithm that finds a schedule with as small a completion time as possible.

\par 问题分析与证明:先考虑最小情况。当只有一个作业时,一定要先在超算上先预处理,然后才能在PC上完成,其总时间$T=p_i+f_i$。
\par 当有两个作业时,假设$J_1$的$p_1$为2,$f_1$为1,$J_2$的$p_2$为3,$f_2$为4。先执行$J_1$,其运行总时间为2+3+4=9。先执行$J_2$,其运行总时间为3+4=7。所以先执行$J_2$。将$J_1$的$f_1$改为5,则先执行$J_1$,其总时间为2+3+4=9,先执行$J_2$,其总时间为3+2+5=10。要先执行$J_1$。
\par 贪心策略为,每次选择在PC上执行最长的执行,每次将$f_i$中最大的放到超算上进行处理,然后将其分配到PC上继续计算。
\par 现考虑一般情况,对上述策略进行正确性证明:首先我们将这$n$个作业按照$f_i$进行降序排序,不妨假设$J_i$的PC执行时间是最长的,则按照贪心策略执行$J_i$。现假设最优结构中不优先执行$J_1$,而执行某一个$J_k(k>1)$,而后再去执行$J_1$。假设执行到n个作业的总时间为$T$,则先执行$J_k$的总时间为max($T+p_k+p_i+f_i, T+p_k+f_k$),由于$f_i$一定大于$f_k$,因此最长时间一定是$T+p_k+p_i+f_i$。先执行$J_i$,总时间为max($T+p_i+p_k+f_k, T+p_i+f_i$),由于$f_i$一定大于$f_k$,所以$T+p_k+p_i+f_i > T+p_i+p_k+f_k$且$T+p_i+p_k+f_i > T+p_i+f_i$,即$T+p_k+p_i+f_i > $max($T+p_i+p_k+f_k, T+p_i+f_i$)。所以无论怎样先执行$J_k$的时间总是比先执行$J_i$的时间长。即每次都需要去先执行在PC上时间最长的。

\begin{algorithm}[h]
    \begin{algorithmic}[0] %每行显示行号
    \caption{Greedy 2}
    \State sort $f_1, f_2,...,f_n$ as f[n] //降序排列$f_1,f_2,...,f_n$
    \State \Return f[n]
    \end{algorithmic}
\end{algorithm}
\par 时间复杂度分析:该算法需要对数组进行排序,其时间复杂度至少为O($log(n)$)。
\end{document}