\documentclass[UTF8,a4paper,12pt]{article}
\usepackage[top=2cm,bottom=2cm,left=2cm,right=2cm]{geometry}
\usepackage{algorithm}
\usepackage{algorithmicx}
\usepackage{algpseudocode}
\usepackage{amsmath}
\usepackage{tikz}

\usepackage{CJK}
\usepackage{amsmath,bm}
\usepackage{indentfirst}
\setlength{\parindent}{2em}
\floatname{algorithm}{PROBLEM}
\renewcommand{\algorithmicrequire}{INPUT:}
\renewcommand{\algorithmicensure}{OUTPUT:}

%\begin{CJK}{UTF8}{gkai}
\title{091M4041H - Assignment Three\\Greedy Algorithm}
\date{\today}
\author{张   帅\\201828018670119\\网络空间安全学院, UCAS}

%\end{CJK}


%\begin{CJK}{UTF8}{gkai}
%	你好,世界
%\end{CJK}


\begin{document}
	\begin{CJK}{UTF8}{gkai}
		\maketitle
	\section{是否存在无向图?}
	\subsection{问题描述}
	
		Given a list of n natural numbers $ d_{1},d_{2},\dots,d_{n} $ , show how to decide in polynomial time whether there exists an undirected graph $ G = (V, E) $ whose node degrees are precisely the numbers $ d_{1},d_{2},\dots,d_{n} $. $ G $ should not contain multiple edges between the same pair of nodes, or “loop” edges with both endpoints equal to the same node.
	
	\subsection{基本思想}
		%由于本题目并没有严格要求构成的图必须是连通无向图,因此不需要考虑图是否连通情况。
		
		假设给定的节点及其度数可以构成图,每次从图中选择一个度数最大的节点,将该节点及所有与其相连的边删除,重复这个过程,那么到最后无向图$ G $必然会变为空。
		
		将度数从大到小排序,每次选取度数最大的点(度数为$ d_{i} $),假设它与随后的$ |d_{i}| $个节点用边连接,将这些边删除,即:随后的$ |d_{i}| $个节点的度数都减1,后面的度数出现-1的情况,说明没有那么多节点与该节点相连,所以就无法构成图;否则,重复该过程,直到遍历完所有的节点为止。
		
		%如果每次都在节点集中任意选择一个节点,则最后的时间复杂度会是指数集的,因此,我们可以设计每次都选择第一个,这样时间复杂度会降到多项式级,并且能保证具有最优子结构性质。在算法上的表述为:在第$ i $次操作中,取出第一个节点,它的度数为$ |d_{i}| $,然后假设它与随后的$ |d_{i}| $个节点相连,那么将$ d_{i} $节点及所有与它相连的边删除,在度数上的表示是:第一个节点$ d_{i} $被删除,随后的$ |d_{i}| $个节点的度数依次减一。如果在第一个节点之后没有$ |d_{i}| $那么多个节点,则表示不能构成无向图,直接返回false,否则重复上面的过程,直到操作到最后,则说明存在这样的无向图。
	\subsection{伪代码}
	
	\begin{algorithm}[htb]
		\caption{Determine whether there exists an undirected graph}
		\begin{algorithmic}[1]
			\Require
			Given a list of n natural numbers $ d_{1},d_{2},\dots,d_{n} $, $ D,n $.
			\Ensure
			Determine whether there exists an undirected graph.
			\Function {IsExistUndirectedGraph}{$ D,n $}
				\State $ flag \gets true $
				\For{$ i = 1 \to n $}
					\State sort D[i:n] in non-increase order					
					\For{$ j = i+1\to i + D_{i} $}
						\If{$ D_{j}-1 != -1$}
							%\State $ D_{i} \gets D_{i} - 1 $
							\State $ D_{j} \gets D_{j} - 1 $
						\Else
							\State \Return $ false $
						\EndIf
						\State $ D_{i} \gets 0 $
					\EndFor
				\EndFor
				\State \Return $ true $
			\EndFunction
		\end{algorithmic}
	\end{algorithm}

	\subsection{贪心选择性质}
		每次选择度数最大的节点删除边,如果该节点都无法满足其度数,那么肯定无法构成图;否则,继续判断。
	\subsection{最优子结构性质}
		从上面的算法思想以及贪心选择中我们可以知道,原问题的解与子问题的解(删去第一个节点及所有与其相连的边的之后的子图)相同。
		
		$$ OPT{D[i\dots n]} = \left\{
		\begin{aligned}
		& OPT(D^{'}[i+1\dots n]) &&\ if\ condition\ 1  \\
		& false	&&\ otherwise
		\end{aligned}
		\right.
		$$
		
		式中$ condition 1 $为:在$ [i+1,n] $节点区域中,存在$ |D[i]| $个节点可以和节点$ i $连通,如果该条件不成立,则返回false。$ D^{'}[i+1\dots n] $为与节点$ i $连通的各个节点度数减1之后的度数列表。
		
	\subsection{正确性证明}
		假设能构成图$ G = (V, E) $,那么依次删除度数最大的节点相连的各条边,删到最后,各个节点的度数必然都为0,所以返回true成立。
		
		而如果存在一个节点的度数要大于子图$ G^{'} = (V-D_{i}, E-E_{D_{i}}) $中节点的个数,则不能构成图。这是因为题目要求,构成的图中,两个节点之间至多只能有一条边,且不存在边的两个端点是同一个节点的情况,所以肯定该节点肯定无法达到其期望的度数,返回false成立。
		
	\subsection{复杂度分析}		
		\begin{itemize}
			\item{\textbf{时间复杂度:}} \\
			排序时间复杂度为为$ T(n)=O(nlogn) $
			
			最好的情况是搜到第一个就出现返回false的条件,时间复杂度为$ T(n)=O(nlogn) $;
			
			最坏的情况即存在一个简单完全无向图,则度数总和为$ n(n-1)/2 $,时间复杂度为:
			\begin{align*}
			T(n) &= n*nlogn+c*n(n-1)/2\\
			&=O(n^{2}logn)
			\end{align*}
			
			而一般情况为$ T(n)=O(n^{2}logn) $
			
			\item{\textbf{空间复杂度:}}\\
			额外使用的空间是常数量级的,即为$ O(1) $。
		\end{itemize}


	\newpage
	\section{作业调度}
	\subsection{问题描述}
		There are n distinct jobs, labeled $ J_{1} , J_{2} , \dots , J_{n} $ , which can be performed completely independently of one another. Each jop consists of two stages: first it needs to be preprocessed on the supercomputer, and then it needs to be finished on one of the PCs. Let’s say that job $ J_{i} $ needs $ p_{i} $ seconds of time on the supercomputer, followed by $ f_{i} $ seconds of time on a PC. Since there are at least n PCs available on the premises, the finishing of the jobs can be performed on PCs at the same time. However, the supercomputer can only work on a single job a time without any interruption. For every job, as soon as the preprocessing is done on the supercomputer, it can be handed off to a PC for finishing.
	
	\subsection{基本思想}
		首先将作业按照$ f_{i} $按照非增序排序,排序之后的作业序列即是作业调度的序列,然后记录每个作业的完成时间,返回完成时间的最大值即可。
		
	\subsection{伪代码}
		\begin{algorithm}[htb]
			\caption{Job Scheduling}
			\begin{algorithmic}[1]
				\Require
					Given two lists $ P $ and $ F $. And the number of jobs, n.
				\Ensure
					Determine the minimum time that overcome these n jobs, m.
				\Function {JobScheduling}{$ P , F , n $}
					\State sort job by $ P $ in non-incresing order
					\State $ m \gets 0 $
					\State $ cur\_sup \gets 0 $ /*current supercomputer's running time */
					\For{$ i = 1 \to n $}
						\State $ cur\_sup \gets cur\_sup + P[i] $
						\State $ m \gets \Call{Max}{m,cur\_sup+F[i]} $
					\EndFor
					\State \Return $ m $
				\EndFunction
			\end{algorithmic}
		\end{algorithm}
		
	\subsection{贪心选择性质}
		每次从工作集中选择$ f_{i} $最大的先执行,然后重复这样的动作。
		
	\subsection{最优子结构性质}
		
		为了方便计算,我们定义$ OPT $由两部分组成,当前的最晚完成时间$ OPT(i) $和当前supercomputer上的运行时间$ CurT(i) $。
	
		\begin{align*}
			CurT(i) = CurT(i-1) +P[i]
		\end{align*}
		$$ OPT(i) = max \left\{
			\begin{aligned}
				& CurT(i)+F[i]\\
				& OPT(i-1)
			\end{aligned}
			\right.
		$$

	\subsection{正确性证明}
	\subsubsection{证明1}
		假设作业$ k $为完成时间最晚的作业,它的完成时间为$ \sum_{i=1}^{k}P[i]+F[k] $,现在证明在作业$ k $后面不可能出现一个作业$ l $,使得$ F[l]>F[k] $。对于作业$ l $,它的完成时间为$ \sum_{i=1}^{l}P[i]+F[l] $。
		
		因为$ k<l $,所以很容易得出$ \sum_{i=1}^{k}P[i] < \sum_{i=1}^{l}P[i] $,又因为$ F[l]>F[k] $,所以$ \sum_{i=1}^{k}P[i]+F[k] < \sum_{i=1}^{l}P[i]+F[l] $,这与作业$ k $为完成时间最晚的作业相矛盾,所以最晚完成的作业的后面不可能存在一个$ F $值更大的作业。
		
	\subsubsection{证明2}
		假设作业序列$ J_{1} , J_{2} , \dots , J_{n} $满足按$ F $值非增序排列,现在按照这个进行作业调度。若此时出现另一个作业$ k $,使得存在存在这种情况:$ F[l]\le F[k]<F[l+1],l\in[1,n] $,将它排在作业调度序列的最后,则其完成时间为$ \sum_{i=1}^{n}P[i]+P[k]+F[k] $。很明显这个值可能会是最大值。现在查找有没有一个序列能够使得这个值变得小一些。
		
		由证明1中可以得出,如果存在$ F[j]<F[k] $,按照$ j,k $进行的作业调度用时要大于按照$ k,j $进行的作业调度用时。而由已知条件$ F[l]\le F[k]<F[l+1],l\in[1,n] $,所以应该将作业$ k $放在作业$ l $之后,作业$ l+1 $之前执行。同理,不应该将作业$ k $放在$ F $值比$ F[k] $大的作业之前执行。
		
		下面考虑$ F $值相等的情况。假设$ F[l]=F[k] $,则按照$ k,l $进行的作业调度最大用时为:$ \sum_{i=1}^{l-1}P[i]+P[k]+P[l]+F[l] $
		按照$ l,k $进行的作业调度最大用时为$ \sum_{i=1}^{l}P[i]+P[k]+F[k]= \sum_{i=1}^{l-1}P[i]+P[l]+P[k]+F[k]$,很明显这两个值相等,所以若两个作业的$ F $值相等,则其调度的先后顺序不影响最大用时的计算。
		
	\subsection{复杂度分析}		
	\begin{itemize}
		\item{\textbf{时间复杂度:}} \\
		排序的时间复杂度为$ O(nlogn) $, 计算最优解的时间复杂度为$ O(n) $,所以总的时间复杂度为:$T(n)=O(nlogn) $
		\item{\textbf{空间复杂度:}}\\
		额外使用的空间是常数量级的,即为$ O(1) $。
	\end{itemize}
	
	\newpage
	\section{是否存在子序列?}
	\subsection{问题描述}
		Given two strings $ s $ and $ t $, check if $ s $ is subsequence of $ t $?
		
		A subsequence of a string is a new string which is formed from the original string by deleting some (can be none) of the characters without disturbing the relative positions of the remaining characters. (ie, ”ace” is a subsequence of ”abcde” while ”aec” is not).

	\subsection{基本思想}
		如果字符串$ s $的长度大于字符串$ t $的长度,则字符串$ s $不可能是字符串$ t $的子序列,返回false。
	
		否则,对字符串$ s $的每一个字母$ s_{i} $,每次都在字符串$ t $的搜索串$ t^{'} $中搜索最靠前的与该字母相同的项$ t_{j} $,如果在搜索串$ t^{'} $中没有找到相同项,则返回false;否则,更新搜索串$ t^{'} $为去掉$ t_{j} $及其前半部分后剩下的子串,继续重复执行该方法,直到字符串$ s $中的每一项都能在字符串$ t $中按序对应为止。
				
	\subsection{伪代码}
		\begin{algorithm}[htb]
			\caption{Is Subsequence?}
			\begin{algorithmic}[1]
				\Require
				Given two strings $ s $ and $ t $
				\Ensure
				check if $ s $ is subsequence of $ t $
				\Function {IsSubsequence}{$ s , t $}
					\State $ n \gets s.length() $
					\State $ m \gets t.length() $
					\If{$ n > m $}
						\State \Return false
					\EndIf
					\State $ i,j \gets 0 $
					\For{$ i = 0 \to n $}
						\While{$ j < m\ \textbf{and}\ s[i] != t[j] $}
							\State $ j++ $\
						\EndWhile
						\If{$ j == m $}
							\State break
						\EndIf
						\State $ j++ $
					\EndFor
					\If{$ i != n $}
						\State \Return false
					\EndIf
					\State \Return ture
				\EndFunction
			\end{algorithmic}
		\end{algorithm}
	
	\subsection{贪心选择性质}
		每次都从字符串$ t $的子串$ t^{'} $中选择最先与$ s_{i} $匹配的项当做字符串$ t $的子序列中的一项。
	\subsection{最优子结构性质}
			
		$$ OPT(s[i\dots n],t[j\dots m]) = \left\{
		\begin{aligned}
			& true &&\ if\ i>n\\
			& OPT(s[i+1\dots n],t[k+1\dots m]) &&\ if\ s[i]==t[k],k\in[j,m]\\
			& false	&&\ otherwise
		\end{aligned}
		\right.
		$$
		第一个式子中,当$ i>n $,意味着字符串$ s $已搜索完毕且其为字符串$ t $的一个子序列;
		
		第二个式子中,$ k $为$ t[j\dots m] $中第一个与$ s[i]$相同的字母的下标;
		
		如果找不到这样的$ k $,则说明字符串$ s $不可能为字符串$ t $的一个子序列。

	\subsection{正确性证明}
		对于$ s[i\dots n],t[j\dots m] $ ,现在从字符串$t[j\dots m] $中寻找第一次$ s[i] $出现的情况,设其下标为$ k,k\in[j,m] $,则再在从字符串$t[k+1\dots m] $中找满足$ s[i+1\dots n] $的子序列。
		假如我们不选第一次$ s[i] $出现的情况,选择第{2,3,or...}次$ s[i] $出现时情况,记下标为$ l $,很明显$ k<l\le m $。这样就需要在字符串$t[l+1\dots m] $中找满足$ s[i+1\dots n] $的子序列。但是可能会出现这样的一种情况,字符串$ s[i+1\dots h],h\in[i+1,n] $只能在$t[k+1\dots l]$找到子序列,而无法在$t[l+1\dots m] $中找到子序列,这样就可能出现错误的结果。
	
	\subsection{复杂度分析}		
		\begin{itemize}
			\item{\textbf{时间复杂度:}} \\
				获取字符串$ s $和$ t $的长度的时间复杂度分别为$ O(n)、O(m) $,检测是否是子序列的过程时间复杂度为$ O(n+m) $,因此总的时间复杂度为$ O(n+m) $。
			\item{\textbf{空间复杂度:}}\\
				额外使用的空间是常数量级的,即为$ O(1) $。
		\end{itemize}
	\end{CJK}	

\end{document} 
